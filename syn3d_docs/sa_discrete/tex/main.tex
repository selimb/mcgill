\documentclass{SelimArticle}
%!TEX root = main.tex

%%%%%%%%%%%%%%%%%%%%%%%%%%%%%%%
% Additional Packages/Options %
%%%%%%%%%%%%%%%%%%%%%%%%%%%%%%%
% \setlist{nosep}
\hypersetup{hidelinks}

%%%%%%%%%%%%%%%%%
% Title Options %
%%%%%%%%%%%%%%%%%
%Add \\[0.3cm] for new line.
\title{SA Discretization}
\author{Selim Belhaouane}
\date{\today}

%%%%%%%%%%%%%%%%%%%%%%%%%
% Additional Formatting %
%%%%%%%%%%%%%%%%%%%%%%%%%
%Horizontal line below section.
\sectionfont{\sectionrule{0pt}{0pt}{-5pt}{0.8pt}}
%Section numbering depth. Value of 2 means numbering ends with subsections.
\setcounter{secnumdepth}{3}
%Table of contents section depth. Same as above.
\setcounter{tocdepth}{2}
\numberwithin{equation}{section}
\numberwithin{figure}{section}

\newcommand{\ra}[1]{\renewcommand{\arraystretch}{#1}}
\begin{document}
\maketitle
% Begin writing here.
\newcommand{\sa}{\ensuremath{\hat{\nu}}}
\newcommand{\diff}[2]{\ensuremath{
    \frac{\partial #1}{\partial #2}
}}
\newcommand{\cmu}{\ensuremath{
    \sqrt{\gamma}\frac{M_\infty}{\mathit{Re}}
}}
\section{Governing Equation}
The Spalart-Allmaras model solves for a variable \sa ~ related to the eddy viscosity
through
\begin{equation*}
    \mu_T = \rho\sa f_{v_1}
\end{equation*}
where
\begin{align*}
    f_{v_1} &= \frac{\chi^3}{\chi^3 + C_{v_1}^3}\\
    \chi &= \frac{\sa}{\nu}
\end{align*}
The governing transport equation is then:
\begin{equation*}
    \diff{\sa}{t} + \mathcal{A} = \mathcal{P} + \mathcal{D} + \mathcal{SD} + \mathcal{FD}
\end{equation*}
where
\begin{align*}
    \text{Advection} :\quad &\mathcal{A} = u_j\diff{\sa}{x_j}\\
    \text{Production} :\quad &\mathcal{P} = C_{b_1}(1 - f_{t_2})\Omega\sa\\
    \text{Destruction} :\quad &\mathcal{D} = \cmu\left\{
        C_{b_1}\left[(1 - f_{t_2})f_{v_2} + f_{t_2}\right]\frac{1}{\kappa^2} - C_{w_1}f_w
    \right\} \left(\frac{\sa}{d}\right)^2
    \\
    \text{Second-Order Diffusion} :\quad &\mathcal{SD} = \cmu \frac{1}{\sigma}\diff{}{x_j}\left[
            \left(\sa + \nu\right)\diff{\sa}{x_j}
    \right]
    \\
    \text{First-Order Diffusion} :\quad &\mathcal{FD} = \cmu \frac{C_{b_2}}{\sigma}
        \diff{\sa}{x_i}\diff{\sa}{x_i}
\end{align*}
The terms are grouped slightly differently than in the original reference because of the common
nondimensionalization factors.

\section{Solving Strategy and Backward Euler Implicit Method}
See discretization of KW-SST for detailed explanations.
The only difference is there is only equation, thus no assumptions
need to be made in the construction of the Jacobian.  However, the notation is described below.

Let $\mathbf{B}_\alpha$ be an operator representing the contribution to the lower diagonal term
of the factorized Jacobian in the $\alpha$ direction. For example, let
$$
\phi = a\sa_{i-1} + b\sa_{i} + c\sa_{i+1}
$$
Then,
$$
\mathbf{B}_\xi(\phi) = \diff{\left(\phi\right)_{i-1,j,k}}{\sa} = a
$$
Similary, let $\mathbf{D}_\alpha$ and $\mathbf{S}$ represent the contributions to the
upper diagonal and diagonal, respectively. Using the same example:
\begin{align*}
    \mathbf{D}_\xi(\phi) &= \diff{\left(\phi\right)_{i+1,j,k}}{\sa} = c\\
    \mathbf{S}(\phi) &= \diff{\left(\phi\right)_{i,j,k}}{\sa} = b
\end{align*}
These variable names have been chosen because that is what is used in \texttt{Syn3D}.

\section{Discretization}
All expansions are done in two dimensions instead of three for the sake of brevity.

%%%%%%%%%%%%%
% ADVECTION %
%%%%%%%%%%%%%

\subsection{Advection}
It should be noted that the advection term appears on the left-hand side of the governing equation,
thus it needs to be \textit{subtracted} from the residual and Jacobian.
\begin{equation*}
    \mathcal{A} = u_j\diff{\sa}{x_j} = u_1\diff{\sa}{x} + u_2\diff{\sa}{y}
\end{equation*}
Transforming to computational space:
\begin{equation*}
    \mathcal{A} = u_1\diff{\sa}{\xi}\diff{\xi}{x} + u_1\diff{\sa}{\eta}\diff{\eta}{x}
        + u_2\diff{\sa}{\xi}\diff{\xi}{y} + u_2\diff{\sa}{\eta}\diff{\eta}{y}
\end{equation*}
Collecting all the $\xi$ (\textit{i} direction) terms:
\begin{equation*}
    \mathcal{A}_\xi = u_1\diff{\sa}{\xi}\diff{\xi}{x} + u_2\diff{\sa}{\xi}\diff{\xi}{y}
\end{equation*}
The advection is discretized using a first-order upwinding scheme
where $q$ determines the flow direction.
$$
q = u\diff{\xi}{x} + v\diff{\xi}{y}
$$
Then, the advection term in the \textit{i} direction can be written as:
$$
\mathcal{A}_\xi = q^+_{i,j} (\sa_{i,j} - \sa_{i-1,j}) + q^-_{i,j}(\sa_{i+1,j} - \sa_{i,j})
$$
where
\begin{align*}
    q^+ &= \frac{1}{2}\left( q + |q| \right)\\
    q^- &= \frac{1}{2}\left( q - |q| \right)
\end{align*}

\subsubsection{Jacobian}
\begin{align*}
    \mathbf{B}_\xi(\mathcal{A}_\xi) &= -q^+_{i,j}\\
    \mathbf{D}_\xi(\mathcal{A}_\xi) &= q^-_{i,j}\\
    \mathbf{S}_\xi(\mathcal{A}_\xi) &= \left(q^+ - q^-\right)_{i,j}
\end{align*}

%%%%%%%%%%%%%%
% PRODUCTION %
%%%%%%%%%%%%%%

\subsection{Production}
$$
\mathcal{P} = C_{b_1}(1 - f_{t_2})\Omega\sa
$$
No discretization is required.
\subsubsection{Jacobian}
\begin{gather*}
    \mathbf{B}_\alpha(\mathcal{P}) = \mathbf{D}_\alpha(\mathcal{P}) = 0 \quad \forall ~ \alpha\\
    \mathbf{S}(\mathcal{P}) = C_{b_1}(1 - f_{t_2})\Omega
\end{gather*}

%%%%%%%%%%%%%%%
% DESTRUCTION %
%%%%%%%%%%%%%%%

\subsection{Destruction}
\begin{gather*}
    \mathcal{D} = K_D\cdot \sa^2\\
    K_D = \cmu\left\{
        C_{b_1}\left[(1 - f_{t_2})f_{v_2} + f_{t_2}\right]\frac{1}{\kappa^2} - C_{w_1}f_w
    \right\} \left( \frac{1}{d}\right)^2
\end{gather*}
Again, no discretization is required.

\subsubsection{Jacobian}
\begin{gather*}
    \mathbf{B}_\alpha(\mathcal{P}) = \mathbf{D}_\alpha(\mathcal{P}) = 0 \quad \forall ~ \alpha\\
    \mathbf{S}(\mathcal{P}) = 2K_D \cdot \sa_{i,j,k}
\end{gather*}

%%%%%%%%%%%%%%%%%%%%%%%%%%
% SECOND-ORDER DIFFUSION %
%%%%%%%%%%%%%%%%%%%%%%%%%%

\subsection{Second-Order Diffusion}
\begin{align*}
    \mathcal{SD} &= \cmu \frac{1}{\sigma}\diff{}{x_j}\left[
            \left( \sa + \nu \right)\diff{\sa}{x_j}
    \right] \\
    &= \cmu\frac{1}{\sigma}\left\{
        \diff{}{x}\left[
            \left( \sa + \nu \right)\diff{\sa}{x}
        \right]
        +
        \diff{}{y}\left[
            \left( \sa + \nu \right)\diff{\sa}{y}
        \right]
        +
        \diff{}{z}\left[
            \left( \sa + \nu \right)\diff{\sa}{z}
        \right]
    \right\}
\end{align*}
For brevity, the constant factor $\cmu \frac{1}{\sigma}$ won't be written anymore throughout
this section.

The terms inside square brackets expand into:
\newcommand{\qdiff}[4]{\ensuremath{
    \diff{#1}{#2}\diff{#3}{#4}
}}
\newcommand{\sanu}{\ensuremath{\left( \sa + \nu \right)}}
$$
    \sanu\diff{\sa}{x_j} = \left( \sa + \nu \right)\left[
        \qdiff{\sa}{\xi}{\xi}{x_j} + \qdiff{\sa}{\eta}{\eta}{x_j}
    \right]
$$

Then:
\begin{align*}
  \diff{}{x}\left[
      \left( \sa + \nu \right)\diff{\sa}{x}
  \right]
  &= \overbrace{
      \qdiff{\xi}{x}{}{\xi}\left[ \sanu \qdiff{\sa}{\xi}{\xi}{x} \right]
    + \qdiff{\eta}{x}{}{\eta}\left[ \sanu \qdiff{\sa}{\eta}{\eta}{x} \right]
  }^{\text{retained}}
  \\
  &+ \underbrace{
       \qdiff{\eta}{x}{}{\eta}\left[ \sanu \qdiff{\sa}{\xi}{\xi}{x} \right]
      + \qdiff{\xi}{x}{}{\xi}\left[ \sanu \qdiff{\sa}{\eta}{\eta}{x} \right]
  }_{\text{ignored}}
\end{align*}
As shown above, terms involving derivatives both in $\xi$ and $\eta$ are neglected. \textit{
My guess is that this is because it makes the ADI formulation impossible, since it would involve
a larger stencil -- like in }\texttt{nsflux.f}\textit{ -- than permitted} It can also be said that
these cross terms have a smaller contribution to the overall diffusion term.

Repeating this above for directions $x, y, z$ and collecting the $\xi$ terms gives:
$$
    \mathcal{SD}_\xi =
      \qdiff{\xi}{x}{}{\xi}\left[ \sanu \qdiff{\sa}{\xi}{\xi}{x} \right]
    + \qdiff{\xi}{y}{}{\xi}\left[ \sanu \qdiff{\sa}{\xi}{\xi}{y} \right]
    + \qdiff{\xi}{z}{}{\xi}\left[ \sanu \qdiff{\sa}{\xi}{\xi}{z} \right]
$$
Then, let:
$$
    \mathcal{SD}_{\xi,x} =
      \qdiff{\xi}{x}{}{\xi}\left[ \sanu \qdiff{\sa}{\xi}{\xi}{x} \right]
$$
We use a flux-like discretization such that:
\newcommand{\diffxix}{\ensuremath{ \left( \diff{\xi}{x} \right)}}
\begin{align*}
    \mathcal{SD}_{xi,x} &= \diffxix_i \left[ F_{i+1/2} - F_{i-1/2} \right]\\
    F_{i+1/2} &= \left( \sanu \qdiff{\sa}{\xi}{\xi}{x} \right)_{i+1/2}\\
\end{align*}
and compute quantities at the face $i+1/2$ using the following:
\begin{align*}
    \phi_{i+1/2} &= \frac{1}{2}\left( \phi_{i+1} + \phi_i \right)\\
    \left(\diff{\phi}{\xi}\right)_{i+1/2} &= \phi_{i+1} - \phi_{i}
\end{align*}
Plugging back into $\mathcal{SD}_{\xi,x}$:
\begin{align*}
    \mathcal{SD}_{\xi,x} = \diffxix_i
& \bigg\{
    \left(\diff{\xi}{x}\right)_{i+1/2} \left[
      \nu_{i+1/2}\cdot (\sa_{i+1} - \sa_i) + \frac{1}{2}(\sa_{i+1} + \sa_i)(\sa_{i+1} - \sa_i)
    \right]
    \\
    & - \left(\diff{\xi}{x}\right)_{i-1/2} \left[
      \nu_{i-1/2}\cdot (\sa_{i} - \sa_{i-1}) + \frac{1}{2}(\sa_{i} + \sa_{i-1})(\sa_{i} - \sa_{i-1})
    \right]
\bigg\}
\end{align*}

\subsubsection{Jacobian}
One can notice that the term $(\sa_{i} + \sa_{i-1})(\sa_{i} - \sa_{i-1})$ is a difference of squares
factorization. Thus, $\mathcal{SD}_{\xi,x}$ can be rewritten as:
\begin{align*}
    \mathcal{SD}_{\xi,x} = \diffxix _i
& \bigg\{
    \left(\diff{\xi}{x}\right)_{i+1/2} \left[
        \nu_{i+1/2}\cdot (\sa_{i+1} - \sa_i) + \frac{1}{2}(\sa_{i+1}^2 - \sa_{i}^2)
    \right]
    \\
    & - \left(\diff{\xi}{x}\right)_{i-1/2} \left[
      \nu_{i-1/2}\cdot (\sa_{i} - \sa_{i-1}) + \frac{1}{2}(\sa_{i}^2 - \sa_{i-1}^2)
    \right]
\bigg\}
\end{align*}
Consequently, the Jacobian contributions of $\mathcal{SD}_{\xi,x}$ are:
\begin{align*}
    \mathbf{B}_\xi(\mathcal{SD}_\xi,x) &=
        \diffxix_i \diffxix_{i-1/2} \bigg\{ \nu_{i-1/2} + \sa_{i-1} \bigg\} \\
    \mathbf{D}_\xi(\mathcal{SD}_\xi,x) &=
        \diffxix_i \diffxix_{i+1/2} \bigg\{ \nu_{i+1/2} + \sa_{i+1} \bigg\} \\
    \mathbf{S}_\xi(\mathcal{SD}_\xi,x) &=
        \diffxix_i \bigg\{
            \diffxix_{i+1/2} ( -\nu_{i+1/2} - \sa_{i})
          + \diffxix_{i-1/2} ( -\nu_{i-1/2} - \sa_{i})
        \bigg\}
\end{align*}

%%%%%%%%%%%%%%%%%%%%%%%%%
% FIRST-ORDER DIFFUSION %
%%%%%%%%%%%%%%%%%%%%%%%%%
\subsection{First-Order Diffusion}
\begin{align*}
    \mathcal{FD} &= \cmu \frac{C_{b_2}}{\sigma} \diff{\sa}{x_i}\diff{\sa}{x_i}\\
                 &= \cmu \frac{C_{b_2}}{\sigma} \left[
        \left(\diff{\sa}{x}\right)^2
        + \left(\diff{\sa}{y}\right)^2
        + \left(\diff{\sa}{z}\right)^2
    \right]
\end{align*}
For brevity, the constant factor $\cmu \frac{C_{b_2}}{\sigma}$ won't be written anymore throughout
this section.

Similar to the other terms, transforming to computational space and collecting the $\xi$ terms
yields:
$$
    \mathcal{FD}_\xi = \left( \diff{\sa}{\xi} \right)_i^2 \cdot \underbrace{\left[
        \left( \diff{\xi}{x} \right)^2
      + \left( \diff{\xi}{y} \right)^2
      + \left( \diff{\xi}{z} \right)^2
    \right]_i}_{COEF}
$$
We then approximate the derivative with a central difference:
$$
    \left( \diff{\sa}{\xi} \right)_i = \frac{1}{2}( \sa_{i+1} - \sa_{i-1} )
$$
Thus:
$$
    \mathcal{FD}_\xi = COEF\cdot \frac{1}{4} \left(
        \sa_{i+1}^2 - 2\sa_{i+1}\sa_{i-1} + \sa_{i-1}^2
    \right)
$$

\subsubsection{Jacobian}
\begin{align*}
    \mathbf{B}_\xi(\mathcal{FD}_\xi) &= COEF\cdot\frac{1}{2}(\sa_{i-1} - \sa_{i+1})\\
    \mathbf{D}_\xi(\mathcal{FD}_\xi) &= COEF\cdot\frac{1}{2}(\sa_{i+1} - \sa_{i-1})\\
    \mathbf{S}_\alpha(\mathcal{FD}_\alpha) &= 0 \quad \forall ~ \alpha
\end{align*}

\end{document}
